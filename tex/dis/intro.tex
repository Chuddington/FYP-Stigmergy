\section{Project Outline} \label{IntroOutline}
"A robotic swarm is composed of a large number of simple physical robots. From
the local interactions between the robots and the interactions of the robots
with the environment, an efficient global intelligence emerges. Multi-robot
coverage is the problem in which a swarm of robots needs to coordinate
decentralised in order to effectively and efficiently cover an unknown
environment. Examples include various monitoring, rescuing, and patrolling
scenarios. The purpose of this project is to set up an experimental
demonstrator, i.e. 'a dark room', in which multi-robot coverage experiments can
be conducted using e-puck robots, implementing the stigmergy principle as
observed in ant colonies. Ants use chemicals, called pheromones, to communicate
with each other via the environment. However, despite of a few reports of using
chemicals in robotic experiments, this is not a straightforward approach due to
difficulties in implementation and limited extendibility. Therefore, we take
advantage of a glow-in- the-dark foil (i.e. a foil covered by phosphorescent
material which absorbs UV light and re-emits the absorbed light at a lower
intensity for up to several minutes after the original excitation). As robots
need to emit light to the glow-in-the-dark foil, each e-puck robot is equipped
with a UV-LED pointing toward the floor. The glowing trails will take up the
role of natural pheromones."

\section{Problems Addressed by the Project} \label{IntroAddressed}
Primarily, the project helps to improve and refine the sharing of information
in conditions which may interfere with direct communication.  By indirectly
communicating, the robots no longer need large amounts of memory to store
the information, as it is held within the environment.  That way, the
information does not need to be stored for long periods of time on the robot,
only for enough time to process what the information means to the robot.

As mentioned in the Project Outline \ref{IntroOutline}, some of the
applications include travelling in potentially uncharted territory, or in
dangerous situations such as searching for explosives whilst patrolling.
There are other fields which this project could benefit from, too.  Some
examples could include the medical profession in surgery, by having machinery
following the lines drawn on patients instead of staying away from the
localised information.  Another potential use could be within biology.  As this
project aims to emulate some of the functional methods that animals have,
successfully doing so can help us to document and understand reasons for the
evolutionary choices that were made for the species.  This project itself does
not focus on these potential applications - instead it is the functionality and
implementation of an aspect that some animals possess.

\section{Aims and Objectives of the Project} \label{IntroAims}
Project aims include coding an implementation of a stigmergic algorithm for the
'e-puck' hardware platform, which will utilise the use of glow-in-the-dark
materials as the method of storing the localised information and processing the
data in such a way that it allows for a uniform spread of entities across an
area.  Plans for an area are also included within the scope, so that
researchers have a basic platform for testing created algorithms.

\section{Challenges within the Project} \label{IntroChallenges}
There are a couple of challenges that this project has that are inherent to the
subjects that link together as a part of the project.  As with anything which
has robotics as it's main focus, it can be difficult to produce workable source
code, as the platform you are writing for is different to the target
architecture, more often than not, and have specific Application Programming
Interfaces (APIs) which need to be utilised.

Another hurdle to overcome for this project would be how the robots are
required to have some form of autonomy.  As their reactions are based on the
environment, meaning that each iteration can have a slightly different outcome,
depending on the criteria.  In larger projects, this can make testing difficult
and time-consuming.

There are also challenges which affect all projects.  Potential risks such as
time management, disorganisation and ambiguity all play a role within the 
final outcome of the project.

\section{The Produced Solution} \label{IntroProducedSolution}
This project contains the source code for an implementation of the StiCo
algorithm, a stigmergic methodology based on identifying a particular type of
trail left by the agents within a robotic swarm, and then moving away from said
trails.  This allows the agents to spread across an area in a uniform fashion,
with each robot focusing on a particular area.  There are also some schematics
for a proposed arena to hold the robots, allowing a testing ground to be built.

\section{The Project's Success} \label{IntroSuccess}
Unfortunately, as a generalisation, the project would be deemed a failure.
The 'Challenges within the Project' \ref{IntroChallenges} section lists an
overview for the reasons as to the cause - more detail will be given throughout
this document.