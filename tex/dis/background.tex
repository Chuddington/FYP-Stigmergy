\section{Background of the Project}
%Who the project is being done for (your supervisor, and any external customer);
The project is being completed for Professor Karl Tuyls in the
department of Computer Science, University of Liverpool.
The Project is also applicable for anyone wishing to build an area in which to
conduct simulations that utilise the E-Puck platform\cite{ePuckSite}.

%What the proposed solution is, how the aim will be achieved.
The initial solution for the project involves using standalone posts, rope and
two large pieces of cloth.  By using the ropes in a similar manner to a boxing
ring, the solution provides flexibility in terms of arena size yet still 
contains the e-puck robots.  The posts will hold up the large pieces of cloth
--- a darker throw for the internal ceiling of the arena, and a lighter cloth
for the theoretical 'roof' of the structure.  The paler cloth reflects most 
external light which would otherwise cover the sectioned
area; the darker colour absorbs light within the structure.

With this solution the size of the area can be modified thanks to the rope.
The dual-layered cover prevents most light from entering the area.
Openings that may be made within the lower layer of fabric would allow
visual monitoring with little compromise.

There are multiple research papers on decentralised robots --- for patrolling,
there is the Edge Ant Walk (EAW) algorithm, which V. Yanovski has worked on
\cite{Yanovski2003}.  Due to memory limitations, the demonstration will be 
the StiCo\cite{Ranjbar-Sahraei2012} as well as the work on HybaCo
\cite{Broecker2015}.  Research into Robotic implementations of pheromone-using
insects will be the main area of
research\cite{Yanovski2003,Ranjbar-Sahraei2012,Broecker2015}.
Tangentially relevant information includes looking into Auction-based
methods of sharing tasks\cite{Schneider2015}.

\section{Existing Solutions}
There has been previous attempts at this project, from the paper 'StiCo in
Action' \cite{Ranjbar-Sahraei2013Demo}.  The project is very similar in
that an implementation of the StiCo algorithm has been designed for both
projects, for the same robots.

There are also variations on the robot that you can use for this test.  It
could be possible to use a Raspberry Pi \cite{raspberryPiSite} with it's
camera attachment as a base for custom made robots.

There's also multiple emulators for the e-puck robot, so the information can
be simulated in the circumstance that you do not have enough robots at the
time, or if it would be inefficient to test with the higher quantities of
robots.  From the simulators listed \cite{ePuckSiteSimulators}, during the
project I have attempted to test 'ENKI' (\cite{enkiSite}) as well as 'v-Rep'
\cite{vRepSite}.  Whilst I was happy with the concept of Enki, I could not
find out enough from the documentation to compile and test my code within this
simulator.  V-Rep, on the other hand, had a lot of information show initially
and I have included some images which show my progress.

\section{Relevant Research}
Information which has been of use within this project would be the website and
documentation of the robotic platform known as the 'e-Puck'
\cite{ePuckSite}, as it was the target platform for the project.  This robot
is ideal as it has the requirements for the project - a light, camera and 
propulsion.  The robot can be controlled externally via Blue-tooth as well as
running a program.

Similar theories, such as the BeePCo and HybaCo algorithms
have also been heavily relevant within the process of this project
\cite{Broecker2015Demo,Caliskanelli2015,Lemmens2008}, and have been used as a
form of unofficial evaluation throughout the project as well as a valuable
source of information in the theory, and it's application, of robotics-based
projects.

There was a lot of programs which I have had little to no experience in using,
which were utilised within the project.  A main example would be the notion
of coding for a platform you are not writing your source code upon.
Personal preference is to code using simpler editors and without large
development environments with multiple perspectives on the source code.  Whilst
much has been realised, my preference still holds.  Another section of software
that was utilised during the project which required research was the
simulators, as I had no prior experience in emulating robotics.

As part of the project, an arena and it's blueprints were erected to test the
concept of using rope as the edges of an arena - reducing the amount of heavy
parts and making a smaller storage footprint.  For this, I had learnt a basic
knot which helps to hold the rope together and increase it's rigidity in
between the connected posts.

\section{Project Requirements}
%What the aim of the project is, what it is intended to achieve;
Project Aims include building the testing grounds for the robotic simulations as
well as showing a demonstration of the arena through the use of the e-Puck
hardware platform.  With the completion of this project, researchers have the
capability of producing a testing ground for their experiments with the e-Puck
hardware platform.  The demonstrative section aims to show that the arena is
capable and performs it's required task.

For the robots, an implementation of the StiCo algorithm will be applied to
multiple e-Pucks for this project.  The program will be coded within the C
Programming language, and will utilise header files which define functionality
for the installed camera, wheels and LED lights whilst the source code will
handle the calling of said functionality, plus the image processing,
reading if there's a light trail ahead of the robot.