%description of anticipated components
\subsection{Project Components} \label{desSysComp}
Anticipated components for the project include documentation for each stage of
development, source code and it's compiled version of the StiCo algorithm for
the e-Puck robotic platform.  The final component will be a constructed arena
for testing the source code with the e-Puck system, along with blueprints that
are provided in the Design phase of the project.

These together will complete the aim of providing a dark room for researching
and testing of algorithms for the e-Puck system, along with a demonstration
program to show whether the constructed arena is successful.

%description of data structures to be used
\subsection{Proposed Data Structures} \label{desSysData}
Currently, it the project will not contain advanced data structures
such as Sets, Queues, Arrays or Stacks.  The program will have basic data-type
variables to store the radius of the circular path the robot will take, a 
variable to check whether the robot has scanned a light path successfully and,
time permitting, an integer variable to be used as a state machine - running
StiCo initially then BeePCo/HybaCo should the user wish the state to change.
The changing of state can be called by pressing a button on the robot, which 
will make the agent to execute the selected algorithm.

%algorithms to manipulate these data structures
\subsubsection{Data Structure Manipulation} \label{desSysDataMan}
With the StiCo algorithm, no user input would be required to manipulate the
data structures in place.  When a light trail is detected, the motor strength
to each wheel will be swapped to allow the robot to turn in the other 
direction.  If the HybaCo algorithm is implemented concurrently, buttons on the
e-Puck system would be used to differentiate between running the StiCo
implementation and the HybaCo algorithm.

%design of interfaces
\subsection{Interface Design} \label{desSysInt}
As the project is using a robotic system, the interface between user and program
is already defined.  Extending the deliverables to include an implementation of
the HybaCo algorithm will mean applying an event-driven stage in the program, to
allow the user to press a button on the e-Puck system to define which algorithm
should run.  This will act as a basic state machine so the user may choose
which algorithm each robot executes. Lights on the robot can be used to indicate
which state has been selected.

%description of evaluation of system 
\label{desSysEval}
Evaluating the Project will fall into two categories:  the constructed dark
room and the implementation of the algorithms, StiCo and potentially HybaCo
too.

The dark room will be evaluated based on how little light gets through the
covering and how the edges can keep the robots from leaving the sectioned area.
Testing for the light levels can be done by looking at the constructed arena to
see how strongly the flooring glows after normalising in the environment.  A 
torch can then be shone into the arena to see whether the floor can successfully
hold the light for an amount of time.

For evaluating the program, the robots need to interact with the light trails
they leave behind in some form to help evaluate the effectiveness of the dark 
room.  This is achieved by implementing the StiCo algorithm, where the robots
will change direction when they come into contact with a light trail.

\subsection{Project Pseudo-code}
%pseudo code of main methods
\begin{algorithm}
  \caption{StiCo Algorithm\cite{Ranjbar-Sahraei2012Demo} }
  \label{desSysPseuStiCo}
  \begin{algorithmic}[1]
  \Require Each robot can deposit/detect pheromone trails \par
  \State Initialise: Choose circling direction (CW/CCW)
  \Loop
    \While{ (no pheromone is detected) }
      \State Circle around
      \State deposit Pheromone
    \EndWhile
    \If{ (interior sensor detects pheromone) }
      \State Reverse the circling direction
    \Else
      \While{ (pheromone is detected) }
        \State Rotate
      \EndWhile
    \EndIf
  \EndLoop
  \end{algorithmic}
\end{algorithm}

\subsection{Arena Design}
The design above is based on two main concepts -- flexibility and ease of use.
Once the glow in the dark flooring is placed, four posts are then clamped to the
edges of the table.  This ensures the flooring does not move and provides
stability for the rope.  Using a sheepshank knot, applied to the inner edges of
the posts as suggested in the top down view \ref{desSysTopImg}, it is 
estimated that to completely fence the arena with one piece of rope it should
approximately be 11 metres.  Using rope means that the arena can be of whatever
size is deemed appropriate, so long as there is enough to sufficiently cover the
generated edges.

Assuming a circular table, four posts, a sheepshank knot and wrapping the rope
around each post twice, the following algorithm calculates the minimum required
length of rope (it is suggested that you round up to the nearest half a metre 
to make sure):

\begin{algorithm}
  \caption{Rope Length Calculator (Measurements in centimetres) }
  \label{desSysRope}
  \begin{algorithmic}[1]
    \State $ Edge = 3(\sqrt{2(r^{2} ) } ) + 2 $
    \Comment Multiplying by 3 is due to tripling the rope over
    \State $ WrappedPost = (postPerimeter \times 2) + 2 $
    \Comment Trailing 2's are for securing the rope
    \State $ TotalRequiredRope = 4(Edge + WrappedPost) $
  \end{algorithmic}
\end{algorithm}