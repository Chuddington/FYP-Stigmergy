\thispagestyle{plain}
\begin{center}
    \textbf{Abstract}
\end{center} \label{dissAbstract}
A robotic swarm is composed of a large number of simple physical robots. From
the local interactions between the robots and the interactions of the robots
with the environment, an efficient global intelligence emerges. Multi-robot
coverage is the problem in which a swarm of robots needs to coordinate
decentralised in order to effectively and efficiently cover an unknown
environment.

The initial solution for the project involves using standalone posts, rope and
two large pieces of cloth.  By using the ropes in a similar manner to a boxing
ring, the solution provides flexibility in terms of arena size yet still 
contains the e-puck robots.

There has been previous attempts at this project, from the paper 'StiCo in
Action' \cite{Ranjbar-Sahraei2013Demo}.  The project is very similar in
that an implementation of the StiCo algorithm has been designed for both
projects, for the same robots.

For the robots, an implementation of the StiCo algorithm will be applied to
multiple e-Pucks for this project.  The program will be coded within the C
Programming language, and will utilise header files which define functionality
for the installed camera, wheels and LED lights whilst the source code will
handle the calling of said functionality, plus the image processing,
reading if there's a light trail ahead of the robot.

This project contains the source code for the StiCo algorithm, a stigmergic
methodology based on identifying a particular type of trail left by the agents
within a robotic swarm, and then moving away from said trails.

For some unknown reason, the IDE was unable to find the location of the header
files required within the program, meaning that the source code was unable to
be compiled, so I could not test the code upon 'live' robots.  Because of this,
I spent the time I had left in working with simulators - notably ENKI and
v-Rep \cite{enkiSite,vRepSite}.

I have used 'GitHub' as my method of storing my project \cite{GithubRepo}, as I
enjoy the portability granted to me and the ease of use it has given me whilst
interacting with this project.