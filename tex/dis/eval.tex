\section{How the Project is evaluated}
The project has a couple of criteria which are broken down between the arena
and code for the target platform.  For the robot, it should be able to
interact with it's surroundings in some way - originally it would have been by
the light left on glow in the dark foil through the use of the on-board camera
and whether or not the robot can pick up and process the images would have been
something to evaluate as well.  However, as the project was later adapted to
interact with sound due to the change from real-world testing into
an evaluation which is simulator-based, using the camera would have been a moot
point.  This can be tested with a single robot by simulating a sound source
move across a robot's area of movement to see if the robot interacts with the
external source.  With real-world robots focusing on light, the same can be
achieved by using a torch to simulate the localised messages.

Another point of evaluation would be whether the robots interact with the
trails left by other light sources, notably other e-Puck robots.  This original
evaluation changed during simulations to become whether the robots react to
other e-Puck devices sending off sound, and changing direction to compensate
depending on the level of noise.  Again, a sound source can be simulated to act
like a robot not being controlled by the StiCo algorithm so it can be seen
whether the robots would be affected.  A person-controlled light source would
have been used during physical testing to obtain the same effect.

For the arena, a couple of main concepts were scrutinised.  In terms of
functionality, the arena would have to be able to contain the robots.  This was
tested by applying force to the edges of the constructed arena and whilst there
was some give in the rope, there was not enough for the robot to slip
underneath the rope and fall off of the table.

Another point of evaluation for the arena would be whether the clamps could
withstand being used as posts, and whether it could hold the glow-in-the-dark
foil underneath them.  Whilst it is untested whether the clamps would hold the
foil in place to reduce slipping whilst in use, the clamps were rigid enough to
hold in place as the rope was being fully connected and tightened, so it could
be assumed that the foil would hold under the clamps.

Primarily, it was self-evaluation that drove the project.  This may not have
been what is best for the project, due to the temperamental nature that
robotics seems to entail.  By having a third party be a part of testing the
project, more rigorous evaluations could have been performed by identifying
different tests that could have been carried out.

\section{A Critical Evaluation}
\subsection{Project Outcome}
Overall, I'm disappointed in the outcome of this project.  I believe that
time could have been saved if I had finished fully setting up the environment
before wanting to have something that could be produced.  If this was
considered beforehand, maybe the project would not have taken a turn into
simulations and other changes due to simulating the robots.  I've enjoyed
creating the data that I have whilst on this project, but I have yet to know
whether the code produced truly works.  I don't think that many problems would
be had, once the header files are located - it also provides a base for others
to work on robotics and artificial intelligence so I can be content with the
work that's been completed, even if the output itself feels underwhelming to
myself.

\subsection{Project Strengths and Weaknesses}
I believe that a strength in this project was the ability to adapt to the
increasing pressures that this project has been a part of.  Without the ability
to simulate, this project may have been in a bad condition.  Whilst there are
more things which can go wrong in a robotics-orientated project, there is also
more flexibility in producing results which can be used - this is not something
I believe that can be easily said for a project that relies on a database.

I believe another strength was the arena concept - the materials are
lightweight and can be easily set up, taken down, stored and carried.  All of
the apparatus can fit in one bag, which makes transporting the arena much 
easier and less cumbersome.

On the other hand, a major weakness within this project was my unfamiliarity
with some of the required tools.  Instead of learning about the MPLAB X IDE, I
chose to use 'gVim' \cite{vimSite} whilst building my source code, and in
hindsight that was detrimental to this project.  A similar thing can be said
for the simulators - I had not learnt anything more than basic techniques in
simulating the robots so had simulation been a focus from the start of the
project, it may have ended with more relevant data.