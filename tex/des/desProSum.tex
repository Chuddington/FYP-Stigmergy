%Statement of background, aims and objectives
"A robotic swarm is composed of a large number of simple physical robots. From
the local interactions between the robots and the interactions of the robots
with the environment, an efficient global intelligence emerges. Multi-robot
coverage is the problem in which a swarm of robots needs to coordinate
decentralised in order to effectively and efficiently cover an unknown
environment. Examples include various monitoring, rescuing, and patrolling
scenarios. The purpose of this project is to set up an experimental
demonstrator, i.e. 'a dark room', in which multi-robot coverage experiments can
be conducted using e-puck robots, implementing the stigmergy principle as
observed in ant colonies. Ants use chemicals, called pheromones, to communicate
with each other via the environment. However, despite of a few reports of using
chemicals in robotic experiments, this is not a straightforward approach due to
difficulties in implementation and limited extendability. Therefore, we take
advantage of a glow-in- the-dark foil (i.e. a foil covered by phosphorescent
material which absorbs UV light and re-emits the absorbed light at a lower
intensity for up to several minutes after the original excitation). As robots
need to emit light to the glow-in-the-dark foil, each e-puck robot is equipped
with a UV-LED pointing toward the floor. The glowing trails will take up the
role of natural pheromones."

The Aims of the Project are to construct a testing arena for the e-Puck robotic 
platform, notably a "dark room," which will allow the usage of the robot's
lights to leave localised messages on flooring which can store and emit light.
Sucessfully fulfilling the aim will mean that other users of the e-Puck system
have a basis to create their own dark room.  This may improve research in swarm
robotics or, if used for demonstration purposes, can bolster interest in
applicable fields.

The primary Objective of the project are to build said arena with dimensions
small enough to fit on a circular table approximately 60 centimetres in radius.
As a secondary Objective, completion of the project will produce a program for
the e-Puck robotic system to demonstrate the effectiveness of the environment the robot will be placed in.  The program will initially be an implementation of
StiCo\cite{Ranjbar-Sahraei2012Demo}, but may have modifications dependant on
time constraints.

%Highlight changes to original specification; include why and justification
The original requirements document stated that the program will be similar to 
StiCo and HybaCo\cite{myReq}.  The language used suggests that the final
program will be a variant of the two algorithms.  This is no longer the case
-- the program will be an implementation of the StiCo algorithm and a separate,
modified program will only be available should there be enough time to
change the original implementation.  This is to allow the time to properly
implement the StiCo algorithm as the compiled code is used as a proof of
concept that the constructed dark room is a viable environment for the testing
of light based communication using the e-Puck system.

%Summary of research and analysis done so far
\subsection{Relevant Research and Analysis} \label{desResAnal}
%summary of what was read, tested (e.g., techissues)? how outcomes affect
%design?
There has been some research on Stigmergic algorithms.  The main algorithm used
within this project will be 
StiCo\cite{Ranjbar-Sahraei2012,Ranjbar-Sahraei2012Demo,Ranjbar-Sahraei2013}.
When implemented, the algorithm can help to reduce the total area of terrain
covered by multiple robots which would improve efficiency and total area 
being patrolled upon.  This algorithm will be used to show that the dark room
that will be built is functional.

BeePCo is another, different algorithmic solution in stigmergic robotics.  
Whilst it is not applicable for the project due to it's reliance on direct
communication between agents and would be a very different implementation 
compared to StiCo, it can be useful with fewer robots to help monitor all of the
area by maintaining distance through network connections.

HybaCo is a combination of both StiCo and BeePCo -- initially running the latter
algorithm whilst a direct connection is available and then switching to StiCo
when connecting to other robots is no longer possible, perhaps due to range
limitations\cite{Broecker2015}.  If time permits, I will modify the code
to implement the BeePCo algorithm, and mold the final result into a HybaCo
implementation.

%any analysis done? and their implications

