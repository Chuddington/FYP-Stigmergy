%description of anticipated components
Anticipated components for the project include documentation for each stage of
development, source code and it's compiled version of the StiCo algorithm for
the e-Puck robotic platform.  The final component will be a constructed arena
for testing the source code with the e-Puck system, along with blueprints that
are provided in the Design phase of the project.  These together will complete
the aim of providing a dark room for researching and testing of algorithms for
the e-Puck system, along with a demonstration program to show whether the 
constructed arena is successful.

%description of data structures to be used
Currently, it is believed that the project will not contain data structures
such as Sets, Queues, Arrays or Stacks.  The program will have basic datatype
variables to store the radius of the circular path the robot will take, a 
variable to check whether the robot has scanned a light path successfully and,
time permitting, an integer variable to be used as a state machine - running
StiCo initially then BeePCo/HybaCo should the user wish the state to change.
The changing of state could be called by pressing a button on the robot before
it starts it's run, for example.

%algorithms to manipulate these data structures
With the StiCo algorithm, no user input would be required to manipulate the
data structures in place.  When a light trail is detected, the motor strength
to each wheel will be swapped to allow the robot to turn in the other 
direction.  If the HybaCo algorithm is implemented, buttons on the e-Puck 
system could be used to differentiate between running the initial StiCo
implementation and the potential implementation of the HybaCo algorithm in a
similar manner to a state machine e.g. 2 button presses could indicate running
the HybaCo algorithm whereas 1 button push will run the StiCo algorithm only.

%design of interfaces
Apart from a connection to a computer to pass compiled files over to the e-Puck
system, there are no initial plans for an interface for the project.  An 
interface will only be considered as required, such as to implement a state
machine to the robots so the user may pick and choose which algorithm each
robot runs.  Should this be the case, the buttons available on the robot will 
be the user interface, where the user pushes a single button to denote which
algorithm should be run.  If a second button is available on the robot, this
will be used to execute the currently selected algorithm.  Lights on the robot
can be used to indicate which state is currently running.

%description of evaluation of system 
\label{desBaseEval}
Evaluating the Project will fall into two categories:  the constructed dark
room and the implementation of the algorithms, StiCo and potentially HybaCo
too.

The dark room will be evaluated based on how little light gets through the
covering and how the edges can keep the robots from leaving the sectioned area.
Testing for the light levels can be done by looking at the constructed arena to
see how strongly the flooring glows after normalising it's charge.  A torch can
then be shone into the arena to see whether the floor can successfully hold the
light for an amount of time.

For evaluating the program, the robots need to interact with the light trails
they leave behind in some form to help evaluate the effectiveness of the dark 
room.  This is achieved by implementing the StiCo algorithm, where the robots
will change direction when they come into contact with a light trail.


%OO-Based Design

%use-case diagram
\begin{figure}[h!]
  \label{desUseCase}
  \begin{tikzpicture}
    \begin{umlsystem}{Schedule}
      \umlusecase[x=-3,name=useState]{Set State}
      \umlusecase[x=3,name=useScan]{Scan Floor}
      \umlusecase[x=-2,y=-2,name=useMove]{Move Around}
      \umlusecase[x=3, y=-2,name=useRotate]{Rotate}
      \umlusecase[x=-3,y=-4,name=useSwap]{Swap Motor Strength}
      \umlusecase[x=3,y=-4,name=useConn]{Connect to other Robots} 
    \end{umlsystem}

    \umlactor[x=-8]{User}
    \umlactor[x=8]{Robot}

    \umlassoc{User}{useState}
    \umlassoc{Robot}{useState}
    \umlassoc{Robot}{useScan}
    \umlassoc{Robot}{useMove}
    \umlassoc{Robot}{useRotate}
    \umlassoc{Robot}{useSwap}
    \umlassoc{Robot}{useConn}
  \end{tikzpicture}
  \caption{Use Case Diagram for proposed software solution}
\end{figure}

%interaction chart
%list of objects, attributes & methods
%pseudo code of main methods
%interface design


%Traditional Design

%data dictionaries
%system boundary diagram
%ER diagram
%logical/physical table structure
%transaction matrix
%pseudo code of main methods
This is the StiCo Pseudocode algorithm\cite{Ranjbar-Sahraei2012Demo}.

This is the BeePCo Pseudocode algorithms\cite{Caliskanelli2015}.

This is the HybaCo Pseudocode algorithm\cite{Broecker2015Demo}.
%interface design
%etc


%Empirical Investigation

%statement of hypotheses to be tested
%description of test data to be used
%experiment design: experiments to be performed, any control to be used
%result analysis, including statistical techniques used
%anticipated conclusions


%Devising New Algorithms

%description of problem to be solved
%existing algorithms of related problem and a critical evaluation
%approach to be used to solve the problem
%how new algorithms analysed; mathematical and experimental analysis
%details of mathematical/experimental analysis


