%Who the project is being done for (your supervisor, and any external customer);
The project is being completed for Professor Karl Tuyls in the
department of Computer Science, University of Liverpool, as the primary
supervisor.  The Project is also applicable for anyone wishing to build an area
in which to conduct simulations that utilise the E-Puck platform\cite{ePuckSite}

%What the aim of the project is, what it is intended to achieve;
Project Aims include building the testing grounds for the robotic simulations as
well as showing a demonstration of the arena through the use of the e-Puck
hardware platform.  With the completion of this project, researchers have the
capability of producing a testing ground for their experiments with the e-Puck
hardware platform.  The demonstrative section aims to show that the arena is
capable and performs it's required task.

%What the proposed solution is, how the aim will be achieved.
The initial solution for the project involves using standalone posts, rope and
two large pieces of cloth.  By using the ropes in a similar manner to a boxing
ring, the solution provides flexibility in terms of arena size yet still 
contains the e-puck robots.  Once properly erected, the posts will hold up the
large pieces of cloth --- one of a darker colour which will be the internal
ceiling of the arena, and a piece of fabric which is lighter in colour to serve
as the theoretical 'roof' of the structure.  The lighter colour will help to
reflect most of the external light which would otherwise cover the sectioned
area, and the darker colour to absorb light within the structure, as well as
absorbing light which manages to pass through the lighter fabric.

This solution allows practitioners to modify the size of the area as there is no
permanent link between the posts, thanks to the rope.  The dual-layered cover
over the 
