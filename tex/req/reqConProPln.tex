%background research - information that will help understand the
  %problem/solution
Research into Robotic implementations of pheromone-using insects will be the 
main area of research\cite{Yanovski2003,Ranjbar-Sahraei2012,Broecker2015}.
Tangentially relevant information includes looking into Auction-based
methods of sharing tasks\cite{Schneider2015}.

%Data Required
  %Ethical data usage
    %Synthetic Data / Real Non Human Data / Real Human Data
  %Ethical use of Human Participants
Information on the hardware and it's usage will be part of the required data,
so that software can be created for the platform.  Currently, there will not be
any real data, both human and non-human, involved within this project.

%Design Stage
  %What design methods will be used / Design Documentation
The design stage will include potential blueprints for the construction of the 
dark room, including materials used, dimensions and functional criteria.
This phase will also have a basic outline of the program in pseudo-code that 
will be written for the implementation stage.  

%Implementation Stage
  %Hardware/Software used
  %Applicable testing
The code will be written in C, and will be compiled for the e-Puck hardware
platform.  Basic testing will include making sure that the robots can
successfully interact with the light trails on the floor.  Later testing will
include the benefits of adding more robots into the area at the same time.

%Risk assessment
  %Challenges / new skills aquired
The main challenge will be creating the dark room - the entire project fails
otherwise.  Coding an algorithm mimicking an insect's pheromone usage may bring
a time constraint.

Skills developed include physical handiwork, coding for a platform that is not a
workstation as well as a greater understanding of artificial intelligence and 
it's implementation in a language which is not logic-based, such as Prolog.